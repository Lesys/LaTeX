\documentclass[a4paper,french,titlepage]{article} \usepackage{knonn_layout}

%Glossaire


%\makeglossaries

\glstoctrue

\makenoidxglossaries

%Lien vers les images


\graphicspath{ {images/} }

%Suppression de l'indentations des paragraphes
\setlength{\parindent}{0pt}

%Commandes persos pour ce document
\newcommand{\dateDebut}{12/02/20} %La date de d\'ebut du projet
\newcommand{\dateFin}{12/02/20} %La date limite du projet / la date à laquelle le document a \'et\'e termin\'e
\newcommand{\titledoc}{Compte Rendu XXX} %Le thème principal du document
\newcommand{\undertitle}{Projet d'étude} %Titre du document, sur quoi il porte
\newcommand{\ICC}{Nom de l'école} %Nom de votre école
\newcommand{\school}{\textit{\textbf{\ICC}}} %Lieu dans lequel le compte rendu est fait
\newcommand{\curversion}{Version \no0.0.1} %La derni\`ere version du document (A METTRE A JOUR LE PLUS SOUVENT POSSIBLE)

%Acteurs

%Titre du document
%Pour un compte rendu de stage / alternance: en haut en gros: "undertitle"
%en dessous: "school" + "where"

%Pour une r\'edaction pour l'\'ecole: en haut en gros: "title" (qui sera le thème du cours)
%en dessous: "undertitle" (qui sera le sujet)
\title{
    \begin{minipage}\linewidth
        \flushright\bfseries
        \Huge \titledoc
        \vskip3pt
        \LARGE \undertitle%\school~-~\where 
    \end{minipage}
}

%Entête
\renewcommand{\headrulewidth}{1pt}
\fancyhead[R]{} %\undertitle} %Affche le titre du document avec l'\'ecole à droite
\fancyhead[C]{Nom de l'auteur} %Affiche l'auteur à gauche
\fancyhead[L]{}
%Pied de Page
\renewcommand{\footrulewidth}{1pt}
\fancyfoot[C]{} %Affiche "rien" au centre
\fancyfoot[L]{\today~- \titledoc} %Affichage date du dernier changement à gauche
\fancyfoot[R]{\curversion~-~Page \thepage} %Affiche la version et le num\'ero de la page à droite



\glsresetall
%Glossaire
\newglossaryentry{wiki}{
	name={Wikipedia},
	description={\gls{wiki} est une encyclopédie universelle, collaborative et multilingue, créée par Jimmy Wales et Larry Sanger le 15 janvier 2001, gérée par wiki dans le site web \href{wikipedia.org}{wikipedia.org}.}
}
%\newglossaryentry{gantt}{
%    name={GANTT},
%    description={Outil utilisé (souvent en complément d'un réseau PERT) en ordonnancement et en gestion de projet et permettant de visualiser dans le temps les diverses tâches composant un projet.}
%}

%\newacronym
%[description={Outil permettant de faire des requêtes en amont et de cr\'eer des processus de traitements pour permettre de faire avancer une chaîne.}]
%{etl}{ETL}{Extract, Transform, Load (Extraction, Transformation et Chargement)}


%D\'ebut du document
\begin{document}
    %Affichage du logo de votre école ou entreprise
%    \begin{figure}
%	\includegraphics[width=0.3\textwidth]{logo.png}
%    \end{figure}

    %Construction du titre
    \maketitle
    %Table des mati\`eres / Sommaire
    \tableofcontents
    %Saut de page
    \newpage

	%Section 1 - Introduction
	\section{Section 1}
		\subsection{Introduction}
			Texte de la première sous-section (en général un paragraphe d'introduction).
		\subsection{Itemize}
			Ici un début de puces:
			\begin{itemize}
				\item Item 1
				\item Item 2
				\item Item 3
				\item Item 4
			\end{itemize}
		\subsection{Types de texte}
				\underline{Texte sousligné}
				\textbf{Texte en gras}

		\subsection{Afficher une image}
			On peut aussi appeler la référence de l'image \ref{Image centrée} et cliquer dessus pour y aller.
			\subsubsection{Image de \gls{wiki} centrée de 20\% de la taille du texte}
			\label{Image centrée}
				\begin {figure}[H]
					\begin{center}
						\includegraphics[width=0.2\textwidth]{centre.png}
					\end{center}
					\caption{Ceci est la description de l'image centrée}
					\hspace{150pt}
				\end{figure}

		\subsection{Tableaux}
			\subsubsection{Tableau centré}
				\begin{center}
					\begin{tabular}{| c | c | c |} %p{5cm}
						\hline
						Colonne 1 & Colonne 2 & Colonne 3\\ \hline\hline
						Première case & Deuxième case & Troisième case\\ \hline
						2.1 case & 2.2 case & 2.3 case\\ \hline
					\end{tabular}
				\end{center}

		\subsection{Listing de code}
			Pour compiler un fichier .tex en format PDF, faites les commandes suivantes:
			\begin{lstlisting}
./makepdf.sh compte_rendu.tex
			\end{lstlisting}
			Vous pouvez également faire des diaporamas en \LaTeX.

			Il y a aussi la possibilité de mettre du code
			\begin{itemize}
				\item En python:
					\begin{lstlisting}[language=python]
import numpy as np
Fe = 10000
amp = 100
Tmax = 0.01
N = int(Tmax * Fe/2)

def somme(a, b):
	return a + b

print(somme(5, 6))
					\end{lstlisting}
				\item En C:
					\begin{lstlisting}[language=C]
void main(int argc, char** argv) {
	char* chaine = "Bonjour";
	printf("La chaine est: %s", chaine);
}
					\end{lstlisting}
				\item En Ruby:
					\begin{lstlisting}[language=Ruby]
class Point
	def initialize(x, y)
		@x = x
		@y = y
	end

	attr_accessor :x, :y
end
					\end{lstlisting}
			\end{itemize}

%	\section{Annexes}
%		\subsection{Annexe 1}
%		\label{Annexe 1}
%			\begin {figure}[H]
%				\begin{center}
%					\includegraphics[width=\textwidth]{image.png}
%				\end{center}
%				\hspace{150pt}
%				\caption{Image de l'annexe 1}
%			\end{figure}
%\newpage
	%Webographie
	\section{Webographie}
		\begin{itemize}
			\item \href{https://wikipedia.org}{\underline{On arrive sur la page de \gls{wiki} si tu me cliques dessus}}
		\end{itemize}
%\newpage
    \printnoidxglossaries
\end{document}
